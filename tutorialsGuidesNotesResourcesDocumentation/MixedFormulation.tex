%===============================================================================
%  Universal LaTeX template for "serious" mathematics
%  -----------------------------------------------------------------------------
%  Features
%    • utf8 input, modern fonts (Latin Modern + math)
%    • micro‑typography, clever quoting, hyperlinks
%    • AMS math, physics, accents, bm, mathtools, siunitx, commutative diagrams
%    • theorem & proof environments, colourful links, automatic sizing
%  Compile with lualatex or xelatex for best results (recommended)
%===============================================================================
\documentclass[11pt,a4paper]{article}

%-------------------------------------------------------------------------------
%  Encoding & fonts
%-------------------------------------------------------------------------------
\usepackage[utf8]{inputenc}    % input encoding (not needed with Lua/XeLaTeX)
\usepackage[T1]{fontenc}       % output font encoding
\usepackage{lmodern}           % Latin Modern fonts (text + math)
\usepackage{microtype}         % micro‑typography (protrusion + expansion)

%-------------------------------------------------------------------------------
%  Geometry + appearance
%-------------------------------------------------------------------------------
\usepackage[a4paper,margin=3.0cm]{geometry}
\usepackage{enumitem}          % better control over lists
\usepackage{xcolor}            % coloured links & accents
\usepackage{graphicx}          % \includegraphics
\usepackage{booktabs}          % nicer tables \toprule, \midrule, \bottomrule

%-------------------------------------------------------------------------------
%  Hyperlinks and references
%-------------------------------------------------------------------------------
\usepackage[colorlinks=true,linkcolor=blue!60!black,
            citecolor=green!40!black,urlcolor=violet]{hyperref}
% \usepackage{nameinlink,cleveref}   % \cref cross‑refs with automatic names

%-------------------------------------------------------------------------------
%  Mathematics packages
%-------------------------------------------------------------------------------
\usepackage{amsmath,amsfonts,amssymb,amsthm}  % core AMS packages
\usepackage{mathtools}          % fixes + extends amsmath (\coloneqq etc.)
\usepackage{physics}            % \qty, \dv, \pdv, bra‑ket notation
\usepackage{bm}                 % bold symbols in math (\bm{\alpha})
\usepackage{siunitx}            % units and numbers: \SI{3.0e8}{\metre\per\second}
\usepackage{tikz-cd}            % commutative diagrams
\usepackage{tikz}               % general TikZ drawings
\usetikzlibrary{calc,arrows.meta,decorations.markings}

%-------------------------------------------------------------------------------
%  Theorem & proof environment setup
%-------------------------------------------------------------------------------
\theoremstyle{plain}
\newtheorem{theorem}{Theorem}[section]
\newtheorem{lemma}[theorem]{Lemma}
\newtheorem{proposition}[theorem]{Proposition}
\newtheorem{corollary}[theorem]{Corollary}

\theoremstyle{definition}
\newtheorem{definition}[theorem]{Definition}
\newtheorem{example}[theorem]{Example}

\theoremstyle{remark}
\newtheorem*{remark}{Remark}

%-------------------------------------------------------------------------------
%  Custom commands
%-------------------------------------------------------------------------------
\newcommand{\R}{\mathbb{R}}
\newcommand{\C}{\mathbb{C}}
\newcommand{\Z}{\mathbb{Z}}
\newcommand{\N}{\mathbb{N}}
\newcommand{\E}{\mathbb{E}}
\newcommand{\PP}{\mathbb{P}}
% \DeclareMathOperator{\grad}{grad}
% \DeclareMathOperator{\curl}{curl}
% \DeclareMathOperator{\diver}{div}
\newcommand{\pr}[1]{\left(#1\right)}
\newcommand{\ag}[1]{\left[#1\right]}
\newcommand{\gp}[1]{\left\{#1\right\}}
\renewcommand{\u}[1]{\boldsymbol{#1}}
%-------------------------------------------------------------------------------
%  Title info
%-------------------------------------------------------------------------------
\title{Mixed formulation for elasticity}
\author{APK, and HK}
\date{\today}

%===============================================================================
\begin{document}
\maketitle
\tableofcontents

%-------------------------------------------------------------------------------

\section{Displacement formulation of Elasticity}
In genral we solve the equations 

\begin{subequations}
\begin{align}
\sigma_{ij,j}&=\rho \ddot{u}_i,\\
\sigma_{ij}&=2\mu \epsilon_{ij}+\lambda \epsilon_{kk}\delta_{ij},\\
\epsilon_{ij}&=\frac{1}{2}\pr{u_{i,j}+u_{j,i}}
\end{align}
\end{subequations}

Stress in terms of displacements read
\begin{align}
\sigma_{ij}&=\mu \pr{u_{i,j}+u_{j,i}}+\lambda u_{k,k}\delta_{ij}
\label{eq:stress}
\end{align}

We have the traction from \eqref{eq:stress} to be 
\begin{align}
\sigma_{ij}n_j&=\mu \pr{u_{i,j}+u_{j,i}}n_j+\lambda u_{k,k}\delta_{ij}n_j,\\
t_{i}&=\mu \pr{u_{i,j}+u_{j,i}}n_j+\lambda u_{k,k}n_i
\end{align}

Combining all the above equations we get
\begin{subequations}
\begin{align}
 \sigma_{ij,j}&=\mu \pr{u_{i,jj}+u_{j,ij}}+\lambda u_{k,kj}\delta_{ij},\\
 \sigma_{ij,j}&=\mu \pr{u_{i,jj}+u_{j,ij}}+\lambda u_{k,kj}\delta_{ij},\\
 &=\mu u_{i,jj}+\mu u_{k,ki}+\lambda u_{k,ki},\\
 &=\mu u_{i,jj}+\pr{\mu+\lambda} u_{k,ki}
  \label{eq:helmholtz}
\end{align}
\end{subequations}

or in co-ordiante free form
\begin{align}
\gp{\text{div}\, \u\sigma}&=\mu \nabla^2 \u{u}+\pr{\lambda+\mu}\nabla\pr{\nabla \cdot \u{u}} 
\end{align}

Thus, the governing equation becomes, 
\begin{align}
\mu u_{i,jj}+\pr{\mu+\lambda} u_{k,ki}+f_i&=\rho \ddot{u}_i, \\
% u_{i}(x_i)&=\hat{u}_i\quad \pr{x_i}\in \Gamma_{g_i}\\
u_{i}(x_i)&=\hat{u}_i\quad \pr{x_i}\in \Gamma_{g_i}\\
\mu \pr{u_{i,j}+u_{j,i}}n_j+\lambda u_{k,k}n_i&=\hat{t}_i\quad on \pr{x_i}\in \Gamma_{h_i} 
\end{align}
% Equation~\cref{eq:helmholtz} is the Helmholtz equation.

%-------------------------------------------------------------------------------
\subsection{Finite Element Formulation}

------------------------------------
\section{Mixed formulation} 


\begin{subequations}
\begin{align}
  \mu \pr{u_{i,jj}+u_{j,ij}}-\frac{2\mu}{3}u_{k,ki}-p_{,i}&=\rho \ddot{u}_{i},\\
  p+K u_{i,i}&=0
\end{align}
\label{eq:MFGeq}
\end{subequations}

The equations 
%-------------------------------------------------------------------------------
\section{Diagrams}

\begin{center}
\begin{tikzcd}
A \arrow[r, "f"] \arrow[d, "g"'] & B \\ C \arrow[ru, "h"']
\end{tikzcd}
\end{center}

%-------------------------------------------------------------------------------
\section{Tables and Figures}

\begin{table}[h]
  \centering
  \begin{tabular}{lcc}
    \toprule
    \textbf{Variable} & \textbf{Value} & \textbf{Unit}\\
    \midrule
    Speed of light $c$ & $3.00\times10^8$ & \si{\metre\per\second}\\
    Planck constant $h$ & $6.626\times10^{-34}$ & \si{\joule\second}\\
    \bottomrule
  \end{tabular}
  \caption{Constants.}
\end{table}

%-------------------------------------------------------------------------------
\appendix
\section{Extra Macros}
Add any further macro definitions here.

%===============================================================================
\end{document}